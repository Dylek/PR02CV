\documentclass[a4paper, 11pt]{article}
\usepackage[polish]{babel}
\usepackage[MeX]{polski}
\usepackage[utf8]{inputenc}
\usepackage[T1]{fontenc}
%\usepackage{times}
\usepackage{graphicx,wrapfig}
%\usepackage{anysize}
%\usepackage{tikz}
%\usetikzlibrary{calc,through,backgrounds,positioning}
\usepackage{anysize}
\usepackage{float}
%\usepackage{stmaryrd}
%\usepackage{amssymb}
%\usepackage{amsthm}
%\marginsize{3cm}{3cm}{3cm}{3cm}
%\usepackage{amsmath}
%\usepackage{color}
%\usepackage{listings}
%\usepackage{enumerate}

\author{Marcin Jędrzejczyk,Paweł Ogorzały}
\newcommand{\HRule}{\rule{\linewidth}{0.5mm}} % Defines a new command for the horizontal lines, change thickness here
\newtheorem{defi}{Definicja}
\begin{document}
	% \noindent -  w tym akapicie nie bedzie wciecia
	% \ indent - to jest aut., ale powoduje ze jest wciecie
	% \begin{flushleft}, flushright, center - wyrownianie akapitu
	% \textbf{pogrubiany tekst}
	% \textit{kursywa} 
	% 					STRONY 
	%  http://www.codecogs.com/latex/eqneditor.php 
	%  http://www.matematyka.pl/latex.htm
	% 
	\begin{titlepage}
	
	
		
		
		
		\center % Center everything on the page
		
		%----------------------------------------------------------------------------------------
		%	HEADING SECTIONS
		%----------------------------------------------------------------------------------------
		
		\textsc{\LARGE Akademia Górniczo-Hutnicza im. Stanisława Staszica w Krakowie}\\[1.5cm] % Name of your university/college
		\textsc{\Large Studio Projektowe \\ Projekt zaliczeniowy}\\[1.5cm]
		%\textsc{\Large Krak�w}\\[0.5cm] % Major heading such as course name
		%\textsc{\large }\\[0.5cm] % Minor heading such as course title
		
		%----------------------------------------------------------------------------------------
		%	TITLE SECTION
		%----------------------------------------------------------------------------------------
		
		\HRule \\[0.4cm]
		{\fontsize{30}{40}\selectfont Gra mobilna Sudoku na telefony z systemem Android}
		\HRule \\[5.5cm]
		
		%----------------------------------------------------------------------------------------
		%	AUTHOR SECTION
		%----------------------------------------------------------------------------------------
		
		% If you don't want a supervisor, uncomment the two lines below and remove the section above

\begin{minipage}{0.4\textwidth}
\begin{flushleft} \large 
\emph{Autorzy:}\\
Marcin \textsc{Jędrzejczyk}\\ 
Paweł \textsc{Ogorzały}
\end{flushleft}
\end{minipage}
~
\begin{minipage}{0.4\textwidth}
\begin{flushright} \large
\emph{Opiekun:}\\
 Dr inż. Maciej \textsc{Szymkat}  % Supervisor's Name
\end{flushright}
\end{minipage} \\[5cm]

		
		%----------------------------------------------------------------------------------------
		%	DATE SECTION
		%----------------------------------------------------------------------------------------
		
		{\large \today}\\[3cm] % Date, change the \today to a set date if you want to be precise
		
		%----------------------------------------------------------------------------------------
		%	LOGO SECTION
		%----------------------------------------------------------------------------------------
		
		%\includegraphics{Logo}\\[1cm] % Include a department/university logo - this will require the graphicx package
		
		%----------------------------------------------------------------------------------------
		
		\vfill % Fill the rest of the page with whitespace
		
	\end{titlepage}
	
	\newpage
	
	\tableofcontents
	\newpage
	
	\listoffigures
	\newpage
	
	
	
	
	\section{Sudoku}
	\subsection{Co to Sudoku}
W Sudoku gra się na planszy o wymiarach 9x9 podzielonej na mniejsze "obszary" o wymiarach 3x3.  Na początku gry niektóre z pól planszy Sudoku są już wypełnione liczbami.  Celem gry jest uzupełnienie pozostałych pól planszy cyframi od 1 do 9 (po jednej cyfrze w każdym polu) przy zachowaniu następujących reguł:
\begin{itemize}
\item Każda cyfra może się pojawić tylko raz w każdym wierszu,
\item Każda cyfra może się pojawić tylko raz w każdej kolumnie,
\item Każda cyfra może się pojawić tylko raz w każdym obszarze.
\end{itemize}
\subsection{Liczba możliwych plansz}
W 2005 matematycy Bertram Felgenhauer z Politechniki w Dreźnie oraz Frazer Jarvis z Uniwersytetu w Sheffield udowodnili, że istnieje 6 670 903 752 021 072 936 960 różnych poprawnych plansz Sudoku. Po utożsamieniu wersji różniących się permutacją cyfr, wierszy, lub kolumn, oraz powstałych przez odbicia i obroty, pozostaje 5 472 730 538 plansz[3]. Ciekawostką jest, że aby rozwiązać Sudoku, potrzeba mieć podanych minimum 17 cyfr w całym diagramie, inaczej rozwiązanie będzie niejednoznaczne[4]. Należy przy tym zaznaczyć, że nie każdy układ 17 cyfr daje jednoznaczne rozwiązanie. Liczba znanych 17-cyfrowych plansz Sudoku dających jednoznaczne rozwiązanie to 49 151[5].
\subsection{Metodyka pracy}
Projekt został zrealizowany w zespole: Paweł Ogorzały i Marcin Jędrzejczyk.\\
W celu zrealizowania projektu wykorzystano środowiska projektowe: Visual Studio 2015 i Unity3D. Kod aplikacji został napisany w języku C\#.\\
Za metodykę pracy wybrano metodę przyrostową.
	
\subsection{Krótki opis}	
	Gra mobilna  zawiera wyjaśnienie zasad Sudoku i krótki tutorial jak grać. Możliwe jest wybranie poziomu trudności planszy. Do wyboru są: łatwy, normalny i trudny poziom rozgrywki. Plansze są generowane na początku każdej gry, dzięki temu zminimalizowane zostaje prawdopodobieństwo rozgrywania dwa razy tej samej planszy.\\

Na ilość uzyskanych punktów przez gracza, wpływa ilość popełnianych pomyłek oraz czas ukończenia rozgrywki. Aplikacja posiada lokalną, dla danego urządzenia, tablicę wyników, z podziałem na poziomy trudności.\\

Podczas gry wyświetlane jest rząd przycisków z cyframi od 1 do 9. W momencie wybrania pola, które chcemy uzupełnić rząd podświetli możliwe uzupełnienia kwadratu 9x9. Ponadto w razie błędnego wypełnienia pola, możemy je wyczyścić.

	
\vfill
\newpage	
\section{Interfejs gracza}
\subsection{Opis elementów GUI}
Screen + opis co elementy przedstawione robią
\subsection{Jak grać}



\vfill
	\newpage
\section{Struktura aplikacji}
\subsection{Moduły}
\subsection{Diagram klas}
\subsection{Algorytm generowania plansz}

\vfill
\newpage
\section{Testy i wdrożenia}
\subsection{Wdrożenie}
Aplikacja została zaprojektowana i zrealizowana w celach niekomercyjnych, dlatego nie została opublikowana w Sklepie Play. Jednakże udostępniliśmy cały projekt poprzez publiczne repozytorium github z licencją opensouce MIT.
\subsection{Testy}
Aplikacja została przetestowana na poniższych modelach telefonów:
\begin{itemize}
\item przykład LG Spirit - Android 6.0 - Gra działała płynnie, nie wykryto błędów uniemożliwiających zakończenie gry.
\item
\end{itemize}

\subsection{Wnioski}
Na podstawie testów i komentarzy osób testujących oraz spełnionych wymagań, projekt można uznać za pomyślnie zakończony.


\vfill
\newpage
\section{Rozwój}
Projekt w przyszłości można rozwinąć o kilka rzeczy:
\begin{itemize}
\item połączenie z mediami społecznościowymi, t.j. facebook, twitter,
\item tabele wyników przechowywaną globalnie na serwerze,
\item system wyświetlający reklamy,
\item system rejestracji użytkowników, dla zapewnienia unikalnych nicków graczy.
\end{itemize}

	
	% latex(szybka kompilacja),bibtex,kompilacja,kompilacja
%\bibliographystyle{plain}% plain/abbrv/alpha
%\bibliography{bibliografia}%plik .bib
\end{document}


